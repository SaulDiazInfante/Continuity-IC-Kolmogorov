\usepackage{lipsum}
\usepackage{amsfonts}
\usepackage{amsmath, amssymb}
\usepackage{graphicx}
\usepackage{epstopdf}
\usepackage{algorithmic}
\DeclareMathOperator{\sech}{sech}
\ifpdf
  \DeclareGraphicsExtensions{.eps,.pdf,.png,.jpg}
\else
  \DeclareGraphicsExtensions{.eps}
\fi
\newcommand{\creflastconjunction}{, and~}
\newsiamremark{remark}{Remark}
\newsiamremark{hypothesis}{Hypothesis}
\crefname{hypothesis}{Hypothesis}{Hypotheses}
\newsiamthm{claim}{Claim}
\headers{Continuity respect to initial conditions}{F. Delgado-Vences, 
S. Diaz-Infante, A. Matsumya}

% Title. If the supplement option is on, then "Supplementary Material"
% is automatically inserted before the title.
\title{
    Continuity with respect to initial conditions for a numerical 
    approximation of Kolmogorov equations.
    \thanks{
        Submitted to the editors DATE.
        \funding{
            This work was funded by the Fog Research Institute 
            under contract no.~FRI-454.
         }
    }
}

% Authors: full names plus addresses.
\author{
    Francisco Delgado-Vences
    \thanks{
    CONACYT-UNAM, 
    Instituto de Matem\'aticas, 
    Sede Oaxaca, M\'exico.}
    (\email{delgado@im.unam.mx})
%    
    \and
    Saul Diaz-Infante,
    CONACYT-Universidad 
    de Sonora, Departamento de Matem\'aticas. 
    Hermosillo, Sonora, M\'exico,
    (\email{saul.diazinfante@unison.mx})
%
    \and
    Alan Matzumiya
    Universidad de Sonora, 
    Departamento de Matem\'aticas,
    Hermosillo, Sonora, M\'exico,
    (\email{alan.matzumiya@gmail.com})
    \footnotemark[3]
}

\usepackage{amsopn}
\DeclareMathOperator{\diag}{diag}

%%%%%%%%%%%%
\newcommand{\cqd}{\hfill$\Box$}
\newcommand{\f}{{\mathcal F}}
\newcommand{\IR}{{\mathbb R}}
\newcommand{\R}{{\mathbb R}}
\newcommand{\IN}{{\mathbb N}}
\newcommand{\ind}{\mbox{\Large$\chi$}}
\newcommand{\tor}{{\mathbb T}}
\newcommand{\G}{{\mathbb G}}
\newcommand{\beq}{\begin{equation}}
\newcommand{\eeq}{\end{equation}}
\newcommand{\bal}{\begin{align}}
\newcommand{\eal}{\end{align}}
\newcommand{\beqn}{\begin{equation*}}
\newcommand{\eeqn}{\end{equation*}}
\newcommand{\baln}{\begin{align*}}
\newcommand{\ealn}{\end{align*}}
\newcommand{\tbar}{\bar t}
\newcommand{\xbar}{\bar x}
\newcommand{\ep}{\epsilon}
\newcommand{\Pb}{\mathbb P}
\newcommand{\Rl}{\mathbb R}
\newcommand{\E}{\mathbb{E}}
\newcommand{\tf}{\mathcal{F}}
\newcommand{\hac}{\mathcal{H}}
\newcommand{\hact}{\mathcal{H}_T}
