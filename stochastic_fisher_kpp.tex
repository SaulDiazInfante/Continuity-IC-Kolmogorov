Let $\mathcal{H} = L^2 (0,1)$. We consider the stochastic Fisher-KPP 
    equation in the interval $[0, 1]$
    \begin{equation}
        \label{eqn:fisher-kpp}
        \begin{aligned}
            d X(t, \xi) &= 
                \left[
                    \nu 
                    \partial_{\xi} ^ 2 X(t, \xi)
                    +
                    X(t, \xi) (1 -X(t, \xi) )
                \right]
                dt
                +
                dW(t, \xi),
            \\
            X(t, 0) &= X(t, 1) =0, \quad t>0, 
            \\
            X(0, \xi) &= x(\xi), \quad x \in 
            \mathcal{H} \ .
        \end{aligned}
    \end{equation}
    
\begin{table}[H]
    \begin{tabular}{rc}
        \toprule
        $h$ 
        &
        $x$
        \\
        \bottomrule
    \end{tabular}
\end{table}
%
\begin{figure}[H]
    \centering
    \caption{
        Numerical Solution of the Fisher-KPP 
        \cref{eqn:fisher-kpp} 
        with initial conditions $x$, $y$ at time
        $t=0$.
     }
    \label{fig:fisher_kpp_approximation_t0}
    \includegraphics[width=\linewidth, keepaspectratio]%
    {StochasticFisherEquation/Approximation_t=0.eps}
\end{figure}
%
\begin{figure}[H]
    \centering
    \caption{
        Likening between two solution with closed 
        initial conditions $x$, $y$
        of the stochastic Fisher-KPP
        \cref{eqn:fisher-kpp}.
     }
    \label{fig:likening_fisher_kpp}
    \includegraphics[width=\linewidth, keepaspectratio]%
    {StochasticFisherEquation/simulation_Approximation.png}
\end{figure}
\paragraph{Likening}
    In \Cref{fig:likening_fisher_kpp} we illustrate the distance between initial
    conditions distance between...
\todo{A paragraph to describe and stress the likening}
\todo{plot the two solutions and make zooms to stress the contrast}
%
\todo{Change linear scale to logy scale}