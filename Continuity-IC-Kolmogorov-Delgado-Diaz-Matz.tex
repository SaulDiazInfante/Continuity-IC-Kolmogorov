\documentclass[onefignum,onetabnum]{siamart190516}
%
\usepackage{lipsum}
\usepackage{amsfonts}
\usepackage{graphicx}
\usepackage{epstopdf}
\usepackage{algorithmic}
\usepackage[square, sort&compress, numbers]{natbib}
\usepackage[utf8x]{inputenc}
\usepackage{bm}
\usepackage{booktabs}
\usepackage{etoolbox}
\usepackage{todonotes}
\usepackage{epstopdf}
\usepackage{siunitx}
\usepackage{amsopn}
\patchcmd{\SetTagPlusEndMark}{$}{}{}{}
\patchcmd{\SetTagPlusEndMark}{$}{}{}{}
\ifpdf
  \DeclareGraphicsExtensions{.eps,.pdf,.png,.jpg}
\else
  \DeclareGraphicsExtensions{.eps}
\fi

% Add a serial/Oxford comma by default.
\newcommand{\creflastconjunction}{, and~}

% Used for creating new theorem and remark environments
\newsiamremark{remark}{Remark}
\newsiamremark{hypothesis}{Hypothesis}
\crefname{hypothesis}{Hypothesis}{Hypotheses}
\newsiamthm{claim}{Claim}
\DeclareMathOperator{\diag}{diag}

%%%%%%%%%%%%
\newcommand{\cqd}{\hfill$\Box$}
\newcommand{\f}{{\mathcal F}}
\newcommand{\IR}{{\mathbb R}}
\newcommand{\R}{{\mathbb R}}
\newcommand{\IN}{{\mathbb N}}
\newcommand{\ind}{\mbox{\Large$\chi$}}
\newcommand{\tor}{{\mathbb T}}
\newcommand{\G}{{\mathbb G}}
\newcommand{\beq}{\begin{equation}}
\newcommand{\eeq}{\end{equation}}
\newcommand{\bal}{\begin{align}}
\newcommand{\eal}{\end{align}}
\newcommand{\beqn}{\begin{equation*}}
\newcommand{\eeqn}{\end{equation*}}
\newcommand{\baln}{\begin{align*}}
\newcommand{\ealn}{\end{align*}}
\newcommand{\tbar}{\bar t}
\newcommand{\xbar}{\bar x}
\newcommand{\ep}{\epsilon}
\newcommand{\Pb}{\mathbb P}
\newcommand{\Rl}{\mathbb R}
\newcommand{\E}{\mathbb{E}}
\newcommand{\tf}{\mathcal{F}}
\newcommand{\hac}{\mathcal{H}}
\newcommand{\hact}{\mathcal{H}_T}

\headers{Continuity with respect to initial conditions .}{%
    F. Delgdo-Vences, % 
    A. Matzumiya and %
    S. Diaz-Infante
}

\title{%
    Initial conditions continuity of a numerical
    approximation for Kolmogorov equations
    \thanks{Submitted to the editors \today}
}
\author{
    Francisco Delgado-Vences
    \footnotemark[1]
    \thanks{
    CONACYT-UNAM, 
    Instituto de Matem\'aticas, 
    Sede Oaxaca, M\'exico.
    (\email{delgado@im.unam.mx})
    }
%
    \and %
    Alan Matzumiya
   \footnotemark[3]
    \thanks{
    Universidad de Sonora
    Departamento de Matem\'aticas,
    Hermosillo, Sonora, M\'exico,
    (\email{alan.matzumiya@gmail.com})
    }
%%    
    \and %
    Saul Diaz-Infante
   \footnotemark[2]
    \thanks{%
    CONACYT-Universidad de Sonora, 
    Departamento de Matem\'aticas. 
    Hermosillo, Sonora, M\'exico,
    (\email{saul.diazinfante@unison.mx})
    }
}

\DeclareMathOperator{\sech}{sech}
\ifpdf
  \DeclareGraphicsExtensions{.eps,.pdf,.png,.jpg}
\else
  \DeclareGraphicsExtensions{.eps}
\fi



\DeclareGraphicsExtensions{.pdf,.png,.jpg}
\floatstyle{boxed}
\restylefloat{figure}

\ifpdf
\hypersetup{
    pdftitle={
        Initial conditions continuity of a numerical
        approximation for Kolmogorov equations
    }
  pdfauthor={Fracisco Delgado-Veces, Alan Matsumya, Saul Diaz-Infante}
}
\fi

%opening

\begin{document}
\maketitle

\begin{abstract}
    We provide theory to characterizes the stability respect to initial 
    conditions of a weak numerical scheme to approximate the solution of a
    particular family of SPDEs. Our approach consists in solving the associated 
    Kolmogorov equation of the underlying SPDE whit a spectral method. We also 
    run numerical experiments to illustrate our results.
%    
%    In this paper we prove that the numerical method proposed in \cite{de-fl}
%    is continuous with respect to the initial condition.
\end{abstract}

\begin{keywords}
    Stability, spectral methods, Kolmogorov equation.
\end{keywords}

\begin{AMS}
    60H10, 65C20, 35Q84
\end{AMS}

\section{Introduction}
    Stochastic Partial Differential Equations (SPDEs) are important tools in
modeling complex phenomena, they arise in many fields of knowledge like
Physics, Biology, Economy, Finance, etc. Develop efficient numerical
methods for simulating SPDEs is very important but also very difficult and
challenging.

    The  Fokker-Planck-Kolmogorov (FPK) equation is a partial differential
equation that describes the time evolution of the probability density function
of the velocity of a particle under the influence of drag forces and random
forces, it is a kind of continuity equation for densities. Citing \cite{da-za}
``parabolic equations on Hilbert spaces appear in mathematical physics to model
systems with infinitely many degrees of freedom. Typical examples are provided
by spin configurations in statistical mechanics and by crystals in solid state
theory. Infinite-dimensional parabolic equations provide an analytic description
of infinite dimensional diffusion processes in such branches of applied
mathematics as population biology, fluid dynamics, and mathematical finance.''.
This kind of equations have been deeply studied in the last years, see for
instance \cite{bo-da-ro, da-fl-ro, da} and the references therein.

    Try to finding analytical solutions of FPK associated with SPDEs results
impractical. Thus, work with efficient and accurate numerical schemes is 
crucial. In this way, the spectral methods play an essential role to obtain 
better schemes\textemdash under certain conditions; this sort of methods are 
more accurate than finite differences of finite elements and need fewer grid 
points. Here the adjective ``better'' would be under accuracy, consistency, 
stability, and other targets properties. In this work, we explore the ability 
of the method reported in \cite{de-fl} to preserve the continuity respect to 
initial conditions. That is, if a given problem satisfies certain regularity 
conditions, then two of its solution remain closed if its initial function 
conditions are close. So, we desire that a numerical method reproduce this 
behavior and if it is the case, we say that an underlying method is stable in 
this context.

    Our main contribution is the characterization of mild conditions to
assure the continuity respect to initial function conditions to a family of
SPDEs and the stability of a regarding weak spectral approximation.

    The stability theory for spectral methods is still under 
construction and is an active research area. We mention the seminal works of
L.N. Trefethen and M.R. Trummer \cite{Trefethen1987}, D. Gottlieb et. al.
\cite{Gottlieb1987a} as reference for the deterministic case, and N. Li, J.
Fiordilino, and X. Feng, \cite{Li2019} A. Lang, A. Petersson,  and A.
Thalhammer, \cite{Lang2017} for its stochastic version.


    This paper is organized as follows. In \Cref{fpk-sect} we review the
Fokker-Plank-Kolmogorov equation associated with SPDEs in
a separable Hilbert space. \Cref{sec:ContinuityRespectToInitialConditions} 
provides conditions to assure stability respect initial conditions and in 
\Cref{sec:NumericalExperiments} we illustrate our
results with numerical experiments.

\section{Kolmogorov equations for SPDEs in Hilbert spaces}\label{fpk-sect}
    Let $\mathcal{H}$ be a separable infinite-dimensional Hilbert space with
inner product $( ,  )_\mathcal{H} $ and norm $\|\cdot\|_\mathcal{H}$. We define
a Gaussian measure $\mu$ with mean zero and nuclear covariance operator
$\Lambda$ with  $Tr(\Lambda)<+\infty$.

    We focus on the following Kolmogorov equation
\begin{equation}
    \label{P1s2.3}
    \frac{\partial u}{\partial t}= \frac{1}{2}Tr(QD^2u)+ \langle Ax, Du
    \rangle_\mathcal{H} + \langle B(x),Du \rangle_\mathcal{H},\qquad x\in D(A).
\end{equation}

    Several authors have proved results on existence and uniqueness of the
solution of the Kolmogorov equations, see for instance Da Prato \cite{da} for a
survey, Da Prato-Debussche \cite{da-de} for the Burgers equation,  Barbu-Da
Prato \cite{ba-da} for the 2D Navier-Stokes stochastic flow in a channel.
%
\subsection{On the Ornstein-Uhlenbeck semigroup}\label{OUS-sect}
    Following \cite{liu},  in $\mathcal{H}$ we define a Gaussian measure $\mu$
with mean zero and nuclear covariance operator $\Lambda$ with
${Tr(\Lambda)<+\infty}$ and since $\Lambda:\mathcal{H}\mapsto \mathcal{H}$ is a
positive definite, self-adjoint operator then its square-root operator
$\Lambda^{1/2}$ is a positive definite, self-adjoint Hilbert-Schmidt operator
on $\mathcal{H}$.

    Define the inner product
$
    ( g, h )_0 := 
    \big( 
        \Lambda^{-1/2}g ,
        \Lambda^{-1/2} h
    \big)_\mathcal{H},
    \quad 
    \hbox{\rm for}
    \quad g, h \in 
    \Lambda^{1/2} 
    \mathcal{H}.
$
Let $\mathcal{H}_0$ denote the Hilbert subspace of $\mathcal{H}$, which is the
completion of $\Lambda^{1/2} \mathcal{H}$ with
respect to the norm $\|g\|_0:= ( g, g )_0^{1/2} $. Then ${\mathcal{H}_0}$ is
dense in $\mathcal{H}$ and the inclusion map
$i:\mathcal{H}_0\hookrightarrow\mathcal{H}$ is compact. The triple
$(i,\mathcal{H}_0,\mathcal{H})$ forms an abstract Wiener space.

    Let
$
    \mathbb{H} = L^2 (\mathcal{H}, \mu)
$
denote the Hilbert space of Borel
measurable functionals on the probability space with inner
product
\[
    \big[ 
        \Phi,
        \Psi
    \big]_\mathbb{H}
    :=
        \int_{\mathcal{H}} 
        \Phi(v)
        \Psi(v)\mu(dv),\quad
        \text{ for } 
        \Phi,\Psi\in\mathbb{H},
\]
and norm $\|\Phi\|_{\mathbb{H}}:=\big [\Phi,\Phi\big ]_\mathbb{H}^{1/2}$.
We choose a basis system $\{\varphi_k\}$ for $\mathcal{H}$.

    A functional $\Phi:\mathcal{H}\mapsto \IR$, is said to be a smooth simple
functional (or a cylinder functional) if there exists a
$C^\infty$-function $\phi$ on $\IR^n$ and $n$-continuous linear functional
$l_1,\ldots,l_n$ on $\mathcal{H}$ such that for
$h\in\mathcal{H}$
\begin{equation}
    \Phi(h)=\phi(h_1,\ldots,h_n)\quad
    \mbox{\rm where}\qquad h_i=l_i(h),\quad i=1,\ldots,n.\label{S1.2}
\end{equation}
The set of all such functionals will be denoted by $\mathcal{S}(\mathbb{H})$.
%
Denote by $P_k(x)$ the Hermite polynomial of degree $k$ taking values in
$\IR$. Then, $P_k(x)$ is given by the following formula
\[
     P_k(x)=\frac{(-1)^k}{(k!)^{1/2}} e^{\tfrac{x^2}{2}}
     \frac{d^k}{dx^k}e^{-\tfrac{x^2}{2}}
\]
with $P_0=1$. It is well-known that $\{P_k(\cdot)\}_{k\in\IN}$ is a complete
orthonormal system for $L^2(\IR,\mu_1(dx))$ with
$
    \mu_1(dx) =
        \tfrac{1}{\sqrt{2\pi}}
        e^{-\tfrac{x^2}{2}} dx
$.
Define the set of infinite multi-index as
\[
    \mathcal{J}=\Big\{\bm{\alpha}=(\alpha_i,i\ge 1)\quad \big|\quad\alpha_i\in
    \IN\cup\{0\},\quad |\bm{\alpha}|:=\sum_{i=1}^\infty
    \alpha_i<+\infty  \Big\}.
\]
For $\bm{n}\in\mathcal{J} $ define the {\it Hermite polynomial functionals}
on $\mathcal{H}$ by
\begin{align}
    \label{s1.2}
    H_{\bm{n}}(h) = \prod_{i=1}^\infty P_{n_i}(l_i(h)),\quad
    h \in \mathcal{H}_0, \quad \bm{n} \in \mathcal{J},
\end{align}
and where
$
    l_i(h) = \langle h,  \Lambda^{-1/2} \varphi_i \rangle_\mathcal{H}, \quad
    i=1,2,\ldots \quad
$
where $P_n(\xi)$ is the usual Hermite polynomial for  $\xi\in\R$ and
$n\in\IN$.
%
\begin{remark}
    Notice that $l_i(h)$ is defined only for $h \in\mathcal{H}_0$. However,
    regarding $h$ as a $\mu$-random variable in
    $\mathcal{H}$, we have $\E\big(l_i (h)\big) = \|\varphi_i \|^2  = 1$ and
    then $l_k (h)$ can be defined $\mu$-a.e. $h \in\mathcal{H}$,
    similar to defining a stochastic integral.

    It is possible to identify the Hermite polynomial functionals
    defined in \eqref{s1.2}, for $h \in\mathcal{H}_0$, as a deterministic
    version of the Wick polynomials defined on the canonical Wiener space.(for
    further details see \cite{im} for instance).
\end{remark}

    We have the following result (See Theorems 9.1.5 and 9.1.7 in Da
Prato-Zabczyk \cite{da-za} or Lemma 3.1 in chapter 9 from Chow \cite{liu}).

\begin{lemma}\label{s1.le1}
        For $h\in\mathcal{H}  $ let $l_i(h)=\langle h,  \Lambda^{-1/2}\varphi_i
    \rangle_\mathcal{H}$, $ i=1,2,\ldots$. The set $\{H_{\bm{n}}\}$ of all
    Hermite polynomials on $\mathcal{H} $ forms a complete orthonormal system
    for $\mathbb{H} $. Hence the set of all functionals are dense in
    $\mathbb{H}$. Moreover, we have the direct sum decomposition:
    $
        \mathbb{H} = \bigoplus_{j=0}^\infty K_j,
    $
    where $K_j$ is the subspace of $\mathbb{H} $ spanned by $\{H_{\bm{n}}:
    |\bm{n}|=j\}$.
\end{lemma}
%
Let $\Phi$ be a smooth simple functional given by \eqref{S1.2}. Then the
Fr\'echet derivatives, $D \Phi = \Phi'$ and $D_2 \Phi = \Phi''$ in
$\mathcal{H}$ can be computed as follows:
\begin{equation}
    \label{s1.3}
    \begin{aligned}
        (D \Phi(h), v)
        &=
            \sum_{k=1}^n \big[\partial_k \phi(h_1,\ldots,h_n)\big]
            l_k(v)\nonumber
        \\
        (D^2 \Phi(h), v)
        &=
            \sum_{j,k=1}^n \big[\partial_j\partial_k
            \phi(h_1,\ldots,h_n)\big] l_j(v) l_k(v),
    \end{aligned}
\end{equation}
for any $u, v \in \mathcal{H}$, where
$\partial_k \phi= \frac{\partial}{\partial h_k} \phi$.
Similarly, for $m > 2$, $D^m \Phi(h)$ is a $m-$linear form on $\mathcal{H}^m$
with inner product $(\cdot,\cdot)_m$.
We have
$
    [D^m\Phi(h) ](v_1 , \cdots, v_m )
        = (D^m \Phi(h), v_1 \otimes \cdots \otimes v_m )_m
$,
for $h, v_1 , \ldots , v_m \in \mathcal{H}$.
Consider the following linear stochastic equation
$$
    du_t=Au_tdt+dW_t \qquad 
    u_0=h\in \mathcal{H}.
$$
Where $A: \mathcal{D}(A) \subset \mathcal{H} \rightarrow \mathcal{H}$ is the
infinitesimal generator of a strongly continuous semigroup $e^{tA}$ in
$\mathcal{H}$. $W_t$ is a $Q$-Wiener process
in $\mathcal{H}$. Chow in \cite[Lemma 9.4.1]{liu} has shown the following
result.
\begin{lemma} \label{lemma-AQ}
    Suppose that $A$ and $Q$ satisfy the following:
    \begin{enumerate}
        \item
            $A:\mathcal{D}(A)\subset \mathcal{H}\rightarrow \mathcal{H}$ is
            self-adjoint and there is $\beta>0$ such that
            \[
                \langle Av,v\rangle_ \mathcal{H}\le -\beta\|v
                \|_\mathcal{H}\quad
                \forall v\in \mathcal{H}.
            \]
        \item $A$ commutes with $Q$ in $\mathcal{D}(A)\subset \mathcal{H}$.
    \end{enumerate}
    Then \eqref{OU} has a unique invariant measure $\mu$ which is a Gaussian
    measure on $ \mathcal{H}$ with zero mean and covariance
    operator
    $
        \Lambda=
            \tfrac{1}{2}Q(-A)^{-1}
            =\tfrac{1}{2}(-A)^{-1}Q
    $.
\end{lemma}

Suppose that $A$ and $Q$ have the same eigenfunctions $e_k$ with eigenvalues
$\lambda_k$ and $\rho_k$ respectively.

    It is well-know (See for instance Da Prato and Zabczyk \cite{da-za}) that
the solution of \eqref{OU} is a time-homogeneous Markov process with transition
operator $P_t$  defined for $\Phi\in\mathbb{H}$ given by
\begin{equation}
    (P_t\Phi)(h)=
        \int_{\mathcal{H}}
             \Phi(v) \mu_t ^ h(dv)
             = \E
             \big[
                \Phi(u_t^h)
             \big].
\end{equation}
    Let $\Phi\in\mathcal{S}(\mathbb{H})$ be a smooth simple functional. By
setting $\varphi_k = e_k$ in \eqref{S1.2}, it takes the form
$
  \Phi(h) = \phi(l_1(h), \cdots, l_n (h)),
$
where $l_k(h) = (h, \Lambda^{-1/2} e_k )$. Define a differential operator
$A_0$
on $\mathcal{S}(\mathbb{H})$ by
\begin{equation}\label{def-A0}
  \mathcal{A}_0
    \Phi(v) = \tfrac{1}{2}Tr [RD^2 \Phi(v)] + \langle Av, D\Phi(v)\rangle
    ,\qquad v \in H,
\end{equation}
which is well defined, since $D\Phi \in D(A)$ and
$\langle Av, D\Phi(v)\rangle = (v, A D \Phi(v))_\mathcal{H}$.

The following results have been proved in \cite{liu}.
\begin{lemma}
    Let $P_t$ be the transition operator as defined by \eqref{OU}. Then
    the following properties hold:
    \begin{enumerate}
     \item
        $P_t : \mathcal{S}(\mathbb{H})\rightarrow  \mathcal{S}(\mathbb{H})$
        for $t \ge 0$.
    \item
        $\{P_t , t \ge 0\}$
        is a strongly continuous semigroup on
        $\mathcal{S}(\mathbb{H})$ so that, for any
        $\Phi \in \mathcal{S}(\mathbb{H})$, we have $P_0 = I$ ,
        $P_{t+s} \Phi = P_t P_s \Phi$, for all $t, s \ge 0$, and
        $\lim_{t\downarrow 0}
        P_t \Phi = \Phi$.
    \item
        $\mathcal{A}_0$ is the infinitesimal generator of $P_t$ so that, for
        each $\Phi\in\mathcal{S}(\mathbb{H})$,
        \[
            \lim_{t\downarrow 0} \tfrac{1}{t}\big(P_t- I\big)\Phi
                = \mathcal{A}_0\Phi.
        \]
    \end{enumerate}
    \hfill $\Box$
\end{lemma}

\begin{lemma}\label{Pt-Her}
        Let $H_n(h)$ be a Hermite polynomial functional given by \eqref{s1.2}.
        Then the following hold:
    \begin{align}
        \mathcal{A}_0 H_{\mathbf{n}}(h) 
            &= -\lambda_{\mathbf{n}} H_{\mathbf{n}}(h),
            \\
        P_t H_{\mathbf{n}} (h)
            &= \exp\{-\lambda_{\mathbf{n}} t\} H_{\mathbf{n}} (h),
    \end{align}
    for any $\mathbf{n}\in\mathcal{J}$ and $h \in H$, where
    $
        \displaystyle
        \lambda_{\mathbf{n}}=\sum_{i=1}^\infty n_i\lambda_i.
    $
\end{lemma}

The following Theorem is a Green formula that we will need forward.
Its proof can be seen, for instance, in \cite[Thm. 3.3, Ch. 9,][]{liu}.

\begin{theorem}\label{green-form}
    Let
    $
        \Phi \in \mathcal{S}(\mathbb{H})
    $ be a smooth simple functional and let
    $\mu\sim N(0,\Lambda)$ be a Gaussian measure in $\mathcal{H}$. Then,
    for any $g,h\in\mathcal{H}$ the following formula holds
    \begin{align}
        \int_{\mathcal{H}} (\Lambda h,D\Phi(v))_{\mathcal{H}}  \mu(dv) =
            \int_{\mathcal{H}} (v,h)_{\mathcal{H}}
            \Phi(v) \mu(dv) \ .
            \label{s2.2.1}
    \end{align}

\end{theorem}

\begin{lemma}
    Assume the conditions for  \Cref{Pt-Her} hold. Then, for any
    $\Phi,\Psi\in \mathcal{S}(\mathbb{H})$, the following Green’s formula holds:
    \begin{align}
         \int_{\mathcal{H}} (\mathcal{A}_0 \Phi)\Psi d\mu &=
             \int_{\mathcal{H}} \Phi(\mathcal{A}_0 \Psi) d\mu=
            -\frac{1}{2}\int_{\mathcal{H}} (QD\Phi, D\Psi) d\mu \ .
    \end{align}

\end{lemma}
By \Cref{s1.le1}, for $\Phi \in \mathbb{H}$, it can be represented as
\begin{align} \label{s1.4}
    \Phi(v) &=
        \sum_{n=0}^\infty \phi_{\mathbf{n}} H_{\mathbf{n}}(v),
\end{align}
where $n = |\mathbf{n}|$ and $\mathbf{n}\in \mathcal{J}$. Notice that we can
think in $\mathbf{n}$ as a vector of $r$ dimension, i.e.
$\mathbf{n}=(n_1,\ldots,n_r)$.
Let $\alpha_{\mathbf{n}} = \alpha_{n_1}\cdots \alpha_{n_r}$ be a sequence of
positive numbers with $\alpha_{\mathbf{n}} > 0$, such that
$\alpha_{\mathbf{n}} \rightarrow \infty$ as $n \rightarrow \infty$.
Define
\begin{align*}
     |||\Phi|||_{k,\alpha} 
        &= \Bigg[ \sum_{\mathbf{n}}
        \big(
            1 + \alpha_{\mathbf{n}}
        \big) ^ k |\ phi_n |^2 \Bigg]^{1/2} ,
    \\
    |||\Phi|||_{0,\alpha} 
        &= |||\Phi|||=\Bigg[ \sum_{\mathbf{n}} |\phi_n|^2
    \Bigg]^{1/2},
\end{align*}
which is $L^2(\mu)$-norm of $\Phi$. For the given sequence
$\alpha = \{\alpha_n \}$, let $\mathbb{H}_{k,\alpha}$ denote
the completion of $\mathcal{S}(\mathbb{H})$ with respect to the norm
$|||\cdot|||_{k,\alpha}$. Then $\mathbb{H}_{k,\alpha}$ is called
a Gauss–Sobolev space of order $k$ with parameter $\alpha$. The dual space of
$\mathbb{H}_{k,\alpha}$ is $\mathbb{H}_{-k,\alpha}$.
From now on, we will fix the sequence $\alpha_{\mathbf{n}} =
\lambda_{\mathbf{n}} $, where $\lambda_{\mathbf{n}} $ is given in
\Cref{Pt-Her}.
We shall simply denote $\mathbb{H}_{k,\alpha}$ by $\mathbb{H}_{k}$ and
$ |||\Phi|||_{k,\alpha}$ by $|||\Phi|||_{k}$.

The following results ensure the existence of an extension for the operator
$\mathcal{A}_0$ to a domain containing $\mathbb{H}_{2}$. Their
proofs can be found in \cite{liu} for instance.

\begin{theorem}\label{the-Pt-A}
    Let the conditions on $A$ and $Q$ in \Cref{lemma-AQ} hold.
    Then \sloppy ${P_t: \mathbb{H}\rightarrow \mathbb{H}}$, for $t \ge 0$, is a
    contraction semigroup with the infinitesimal generator $\tilde{A}$.
    The domain of $\tilde{A}$ contains $\mathbb{H}_{2}$ and we have
    $\tilde{A} =\mathcal{A}_0 $ in $\mathcal{S}(\mathbb{H})$.

\end{theorem}

\begin{theorem}\label{the-Pt-A1}
    Let the conditions for \Cref{the-Pt-A} hold true. The differential
    operator $\mathcal{A}_0 $  defined by \eqref{def-A0} in
    $\mathcal{S}(\mathbb{H})$ can be extended to be a self-adjoint
    linear operator $A$ in $\mathbb{H}$ with domain $\mathbb{H}_{2}$.

\end{theorem}
%
    Since both $\tilde{A}$ and $A$ are extensions of $\mathcal{A}_0 $ to a
    domain containing $\mathbb{H}_{2}$, they must coincide there.

    Given the Gauss-Sobolev space $\mathbb{H}_{k}$ with norm
$|||\cdot|||_{k} $ we denote its dual space by $\mathbb{H}_{-k}$ with norm
$|||\cdot|||_{-k}$. Thus, we have the inclusions,
$
    \mathbb{H}_{k} \subset \mathbb{H} \subset \mathbb{H}_{k}.
$
We denote the duality between $ \mathbb{H}_{k}$ and $ \mathbb{H}_{-k}$ by
$
    \langle
        \langle
            \Psi,\Phi
        \rangle
    \rangle_k,
    \qquad
    \Phi\in\mathbb{H}_{k},
    \quad \Psi\in\mathbb{H}_{-k} \ .
$
We also set $\mathbb{H}_{0}= \mathbb{H}$,
with $|||\cdot|||_{0}= |||\cdot|||$ and
$
    \langle
        \langle
            \cdot,\cdot
        \rangle
    \rangle_1=
    \langle
        \langle
            \cdot,\cdot
        \rangle
    \rangle
$,
$
    \langle
        \langle
            \cdot,\cdot
        \rangle
    \rangle_0 = [\cdot,\cdot]
$.
\subsection{A non linear Kolmogorov equation}
Consider the following \sloppy Kolmogorov equation,
%
\begin{equation}
    \label{P1s3.1}
    \begin{aligned}
        \frac{\partial}{\partial t}
            \Psi(v,t)
            &=
               \mathcal{A}\Psi(v,t)
                +
                \langle
                    B(v),D\Psi(v,t)
                \rangle_\mathcal{H},
                \quad \text{a.e. }
                v\in\mathbb{H}_2,
        \\
        \Psi(v,0)
            &=
                \phi(v)\nonumber \ ,
    \end{aligned}
\end{equation}
%
where, as defined in \Cref{the-Pt-A},
$\mathcal{A}:\mathbb{H}_2 \to \mathbb{H}$ is given by
%
\begin{equation}\label{def-A}
    \mathcal{A} \Phi =
        \tfrac{1}{2} Tr[RD^2 \Phi(v)]
        +
        \langle
            A v, D\Phi(v)
        \rangle \ .
 \end{equation}
Hypothesis on $B$ will be specified latter. For now, we will consider that it
is a locally Lipschitz function. The additional term
$ \langle B(v),D\Psi(v,t) \rangle_\mathcal{H}$ is defined $\mu$-a.e.
$v\in\mathbb{H}_2$. We will allow the initial datum $\phi$ will be
in $\mathbb{H}$.

We will study a mild solution of the equation \eqref{P1s3.1}. Let $\lambda>0$
be a parameter. By changing $\Psi$ to $e^{\lambda t}\Psi$ in
\eqref{P1s3.1} we get the following equation:
\begin{align*}
    \label{P1s3.2}
    \frac{\partial}{\partial t}\Psi(v,t)
        &= \mathcal{A}_\lambda\Psi(v,t)
        +
        \langle
            B(v),D\Psi(v,t)
        \rangle_\mathcal{H},
        \qquad \text{ a.e. }
        v \in \mathbb{H}_2,
    \\
    \Psi(v,0)
        &= \phi(v) \ ,
\end{align*}
where $\mathcal{A}_\lambda=\mathcal{A}-\lambda I$, with $I$ the identity
operator in $\mathbb{H}$. Clearly, the problems \eqref{P1s3.1} and
\eqref{P1s3.2} are equivalent, as far for the existence and uniqueness
questions are concerned. We will work on the problem \eqref{P1s3.2}.

    Denote by $P_t$ the semigroup with infinitesimal generator
$\mathcal{A}_\lambda$. The existence of $P_t$ is ensured by the
\Cref{the-Pt-A}.
Then, we can rewrite the equation \eqref{P1s3.2} in an integral form by using
the semigroup $P_t$
\begin{equation}
    \Psi(v,t)=
        e^{-\lambda t} (P_t\phi)(v)
        +
        \int_0^t  e^{-\lambda(t-s)}[P_{t-s}(B,D\Psi_s)](v) ds,
\end{equation}
where we denote $\phi=\phi(\cdot)$ and $\Psi_s=\Psi(\cdot,s)$.
Chow \cite{liu} had proved the following lemma.

\begin{lemma}\label{Lemma.s3.1}
    Let $\Psi\in L^2((0,T);\mathbb{H})$ for some $T>0$. Then, for any
    $\lambda>0$ there exists $C_\lambda>0$ such that
     \begin{align}
        |||\int_0^t e^{-\lambda (t-s)} P_{t-s} \Psi_{s} ds |||^2 \le C_\lambda
        \int_0^T |||\Psi_s|||_{-1}^2 ds,
        \qquad 0< t\le T
        \ .
        \label{s3.1.0}
     \end{align}
\end{lemma}

    We now prove the following theorem on existence and uniqueness of a mild
    solution to \eqref{P1s3.2}.
\begin{theorem}
    \label{Th-EU}
    Suppose that $B:\mathcal{H}\rightarrow \mathcal{H}_0$ satisfies
    $
        (B,D\Phi)\in
        L^2((0,T);\mathbb{H})
    $ for any $\Phi\in\mathbb{H}$ and
    \begin{align}
        \sup_{v\in \mathcal{H}} ||\Lambda^{-1/2}B(v)||_{\mathcal{H}}
            <+\infty.
    \end{align}
    Then, $B$ satisfies
    \begin{align}
        ||| \big(B(v),D\Phi(v) \big) |||_{-1}^2 \le C |||\Phi(v) |||^2  \qquad
        \mbox{for any } \Phi\in \mathbb{H},
        \quad v\in \mathbb{H}_2 ,
        \label{s3.1.1}
    \end{align}
    for some $C>0$.
    Moreover, for $\Phi\in \mathbb{H}$, the initial-value problem
    \eqref{P1s3.2} has a unique mild solution
    $\Psi\in C((0,T); \mathbb{H})$.
\end{theorem}

    For the part of the existence and uniqueness of the solution we will adapt
    the proof of the Theorem 5.2 in Chapter 9 from \cite{liu}.
\begin{proof}
    First we will prove \eqref{s3.1.1}. We have
    \begin{align*}
        ||| \big(B(v),D\Phi(v) \big) |||_{-1}^2 =\sum_{\mathbf{n}}
        (1+\lambda_{\mathbf{n}})^{-1}|\phi_n|^2 ,
    \end{align*}
    with
    \begin{align}
        \phi_n=
        \Big(
            (B(v),D\Phi(V))_{\mathcal{H}}, H_{\mathbf{n}}(v)
        \Big)_{\mathbb{H}}
        =\int_{\mathcal{H}} (B(v),D\Phi(v))_{\mathcal{H}} H_{\mathbf{n}}(v)
        \mu(dv).\label{s3.1.2}
\end{align}

By the \Cref{green-form}, in particular \eqref{s2.2.1}, we have
\begin{align*}
 \int_{\mathcal{H}} (\Lambda h,D\Phi(v))_{\mathcal{H}}  \mu(dv) =
\int_{\mathcal{H}} (v,h)_{\mathcal{H}} \Phi(v)  \mu(dv) ,
\end{align*}
for all $\Phi\in \mathcal{S}(\mathbb{H})$, $g,h\in \mathcal{H}$ and $\mu\sim
N(0,\Lambda)$. Then, in particular, in each direction $H_{\mathbf{n}}$
this formula is still true, so we have
\begin{align*}
 \int_{\mathcal{H}} (\Lambda h,D\Phi(v))_{\mathcal{H}} H_{\mathbf{n}}(v)
\mu(dv) =
 \int_{\mathcal{H}} (v,h)_{\mathcal{H}} \Phi(v) H_{\mathbf{n}}(v) \mu(dv) \ .
\end{align*}
Then, applying this last equality to \eqref{s3.1.2} we get
\begin{align}
 \phi_n &=\int_{\mathcal{H}}
\big(\Lambda[\Lambda^{-1}B(v)],D\Phi(v)\big)_{\mathcal{H}} H_{\mathbf{n}}(v)
\mu(dv)\nonumber\\
 &= \int_{\mathcal{H}} \big(\Lambda^{-1}B(v),v \big)_{\mathcal{H}} \Phi(v)
H_{\mathbf{n}}(v) \mu(dv)\nonumber\\
 &= \int_{\mathcal{H}} \big(\Lambda^{-1/2}B(v),\Lambda^{1/2}v
\big)_{\mathcal{H}} \Phi(v) H_{\mathbf{n}}(v) \mu(dv) \nonumber
    \ .
\end{align}
Thus,
\begin{equation}
    \label{s3.1.3}
    \begin{aligned}
        |\phi_n |^2 &=
            \Bigg|
                \int_{\mathcal{H}}
                    \big(
                        \Lambda^{-1/2}B(v),\Lambda^{1/2}v
                    \big)_{\mathcal{H}} \Phi(v)
                    H_{\mathbf{n}}(v) \mu(dv)
            \Bigg|^2
            \\
            &\le
            \int_{\mathcal{H}}
            \big|
                \big(
                    \Lambda^{-1/2}B(v),\Lambda^{1/2} v
                \big)_{\mathcal{H}}
            \big|^2
            \big|
                H_{\mathbf{n}}(v)
            \big|^2 \mu(dv)
            \int_{\mathcal{H}}
            \big|
                \Phi(v)
            \big|^2  \mu(dv)
            \ .
     \end{aligned}
\end{equation}
We now focus on the first integral. Let $I_1$ be the first integral of
\eqref{s3.1.3}. Then,
\begin{align*}
    I_1
        &
            \le \int_{\mathcal{H}} \big| \big| \Lambda^{-1/2} B(v)
            \big| \big|_{\mathcal{H}}^2
            \big| \big| 
                \Lambda^{1/2}v 
            \big|\big|_{\mathcal{H}}^2
            \big|H_{\mathbf{n}}(v)\big|^2 \mu(dv)
            \\
        &
            \le \sup_{v\in\mathcal{H}} 
            \big|\big|\Lambda^{-1/2}B(v)
            \big|\big|_{\mathcal{H}}^2
            \int_{\mathcal{H}} 
            \big|\big|
                \Lambda^{1/2}v 
            \big|\big|_{\mathcal{H}}^2
            \big|H_{\mathbf{n}}(v)\big|^2 \mu(dv)
            \\
        & 
            \le C \int_{\mathcal{H}} 
            \big|\big| v \big|\big|_{\mathcal{H}}^2
            \big|H_{\mathbf{n}}(v)\big|^2 \mu(dv)
        \\
        & 
            \le C \ .
\end{align*}
%
The last inequality follows by using proposition 3.11 in page 64 from
\cite{da1}. Then, by using this bound on \eqref{s3.1.3} we have.
\begin{align*}
    | \phi_n |^2 &\le C \int_{\mathcal{H}}\big|  \Phi(v)\big|^2  \mu(dv)\\
    &\le C |||\Phi(v) |||^2  .
\end{align*}
Thus,
\begin{align*}
    ||| \big(B(v),D\Phi(v) \big) |||_{-1}^2 
        &\le C |||\Phi(v) |||^2
        \sum_{\mathbf{n}} (1+\lambda_{\mathbf{n}})^{-1}  
        \le C |||\Phi(v) |||^2 ,
\end{align*}
 which proves \eqref{s3.1.1}.

We now prove the existence and uniqueness of a solution to the initial- value
problem \eqref{P1s3.2}. Let $\mathbb{X}_T$ denote the
Banach space $\mathcal{C}([0,T];\mathbb{H})$ with the sup-norm
 \begin{equation*}
  |||\Psi |||_T := \sup_{0\le t\le T} |||\Psi ||| \ .
 \end{equation*}
In $\mathbb{X}_T$ define the linear operator $\mathbb{Q}$ as
\begin{equation*}
 \mathbb{Q}\Psi= e^{-\lambda t}P_t\Phi + \int_0^t e^{-\lambda
(t-s)}P_{t-s}(B,D\Psi_s) ds,\quad\mbox{for any } \Psi\in\mathbb{X}_T.
\end{equation*}
By \Cref{the-Pt-A} $P_t$ is a contraction semigroup, then using this
fact and Lemma \ref{Lemma.s3.1} we have
\begin{align*}
 ||| \mathbb{Q}\Psi|||^2 &\le 2\Bigg[ ||| e^{-\lambda t}P_t\Phi|||^2 +
|||\int_0^t e^{-\lambda (t-s)}P_{t-s}(B,D\Psi_s) ds |||^2\Bigg]\\
 &\le 2\Big[ |||\Phi|||^2 + C_\lambda \int_0^t |||(B,D\Psi_s) |||_{-1}^2 ds
\Big] \\
 &\le 2 |||\Phi|||^2 + C_1 \int_0^t |||\Psi_s|||^2 ds,
\end{align*}
for some $C_1>0$. Hence,
$
 ||| \mathbb{Q}\Psi|||_T \le C(1+||| \Psi|||_T),
$
with $C=C(\Phi,\lambda, T)$. Then, the map $\mathbb{Q}:\mathbb{X}_T
\rightarrow \mathbb{X}_T$ is well defined. We now show that is a contraction
for a
small $t$. Let $\Psi,\Psi'\in \mathbb{X}_T$. Then
\begin{align*}
 ||| \mathbb{Q}\Psi- \mathbb{Q}\Psi'|||^2 &=  |||\int_0^t e^{-\lambda
(t-s)}P_{t-s}\big[(B,D\Psi_s)-(B,D\Psi_s')\big] ds |||^2\\
 &\le  C_\lambda \int_0^t |||(B,D\Psi_s-D\Psi') |||_{-1}^2 ds \\
 &\le  C_2 \int_0^t |||\Psi_s-\Psi'|||^2 ds.
\end{align*}
For some $C_2>0$. It follows that
$%\begin{align*}
 ||| \mathbb{Q}\Psi- \mathbb{Q}\Psi'|||_T  \le  \sqrt{C_2 T}
|||\Psi_s-\Psi'|||_T.
$%\end{align*}
Then, for small $T$, $\mathbb{Q}$ is a contraction on $\mathbb{X}_T$. Hence
the Cauchy problem \eqref{P1s3.2} has a unique mild solution.

\end{proof}


 We now prove a theorem on the dependence on initial conditions for the mild
solution of \eqref{P1s3.2}.
 \begin{theorem}\label{Cont-Mild-Sol}
Suppose that $B:\mathcal{H}\rightarrow \mathcal{H}_0$ satisfies $(B,D\Phi)\in
L^2((0,T);\mathbb{H})$ for any $\Phi\in\mathbb{H}$ and
 \begin{align}\label{s2.22}
  \sup_{v\in \mathcal{H}} ||\Lambda^{-1/2}B(v)||_{\mathcal{H}}<+\infty.
 \end{align}
 Then, the unique mild solution $\Psi\in C((0,T); \mathbb{H})$ for
\eqref{P1s3.2} depends continuously on the initial conditions.

 \end{theorem}
\begin{proof}
We know, with the assumption \eqref{s2.22}, that the existence of a unique
mild solution for \eqref{P1s3.2} is guaranteed by the Theorem \ref{Th-EU}. We
will denote by $\Psi_t^\varphi$ its mild solution at time $t$ with initial
condition $\varphi$:
\begin{align*}
\Psi_t^\varphi=e^{-\lambda t} P_t  \varphi+  \int_0^t e^{-\lambda
(t-s)}P_{t-s}(B,D\Psi_s) ds \ .
\end{align*}
Then,
\begin{align*}
    \Psi_t^\varphi-\Phi_t^\psi
        &=  e^{-\lambda t} P_t \varphi -e^{-\lambda t} P_t
        \psi +  
        \int_0^t e^{-\lambda (t-s)}P_{t-s}(B,D\Psi_s^\varphi-D\Phi_s^\psi) ds
        \\
        &= e^{-\lambda t} P_t (\varphi - \psi) 
        + \int_0^t
            e^{-\lambda(t-s)} P_{t-s} (B, D \Psi_s^\varphi - D\Phi_s^\psi) ds.
\end{align*}
From this expression we get
\begin{align*}
\| | \Psi_t^\varphi-\Phi_t^\psi\| |^2&\le \| |e^{-\lambda t} P_t (\varphi -
\psi)\| |^2+\| | \int_0^t e^{-\lambda
(t-s)}P_{t-s}(B,D\Psi_s^\varphi-D\Phi_s^\psi) \| |^2ds\\
&\le  \|  |\varphi - \psi\| |^2+ C_\lambda\int_0^t  \|
|(B,D\Psi_s^\varphi-D\Phi_s^\psi) \| |_{-1}^2 ds\\
&\le  \|  |\varphi - \psi\| |^2+ C_2\int_0^t  \| |\Psi_s^\varphi-\Phi_s^\psi
\| |^2 ds.
\end{align*}
Thus, by Gronwall's inequality we obtain
\begin{align}\label{s2.23}
    \| |
         \Psi_t^\varphi-\Phi_t^\psi\| |^2&\le \exp(C_2t)  \|  | \varphi - 
         \psi
    \| |^2, 
\end{align}
which implies,
$
\| | \Psi_t^\varphi-\Phi_t^\psi\| |\le \exp(Ct)  \| | \varphi - \psi\| |.
$
This completes the proof.
\end{proof}

\section{Numerical stability respect to  initial conditions}
\label{sec:ContinuityRespectToInitialConditions}

    In this section, we prove the continuity with respect to the initial
conditions for a numerical approximation of the Kolmogorov equation associated 
with an SPDE. Here we understand that a  numerical scheme is stable respect to 
initial conditions if this method reproduces the same behavior when the 
continuous problem satisfies continuity respect initial conditions.

We use \Cref{s1.le1} to write the solution $\Psi_t^\varphi$ as in a
Fourier-Hermite decomposition:
\begin{align}
    \Psi_t ^ \varphi=
        \sum_{\bm{n}\in \mathcal{J}}
        u_{\bm{n}}(t) H_{\bm{n}}(x),
        \qquad
        x \in \mathcal{H},
        \quad t \in [0,T] \ .
        \label{num-approx}
\end{align}

Note that the time-dependent coefficients $ u_{\bm{n}}(t)$ depend on the
functional and on the initial condition but it is not a function of
the initial condition. First we  prove an auxiliary result.

\begin{lemma}\label{le-s3-1}
 Set $\{P_k(\xi)\}_{k\in \IN}$ the family of normalized Hermite polynomials in
$\IR$. For every $k\in\IN$ and $\xi,
 \eta\in\IR$ such that
 $\eta<\xi$ we have that
 \begin{equation}
  P_k(\xi)-P_k(\eta)= C(k)  Pe_{k+1}(\gamma) \cdot (\xi-\eta), \label{Pk-dif}
 \end{equation}
where $\gamma\in (\eta,\xi)$ and $C(k)=\frac{(-1)^k}{(k+1)(k!)^{1/2}} $.
Moreover, $Pe_k(x)$ is the
unnormalized Hermite polynomial of $k$ degree.
\end{lemma}
\begin{proof}
 We know that
 $
     P_k(\xi)=\frac{(-1)^k}{(k!)^{1/2}} e^{\xi^2/2}\frac{d}{d\xi^k}e^{-\xi^2/2}.
 $
 Set $c(k)=(-1)^k (k!)^{-1/2}$, then
\begin{align*}
    P_k(\xi)-P_k(\eta)
        &=c(k)
        \Big[
            e^{\xi^2/2}
            \frac{d}{d\xi^k} e^{-\xi ^ 2/2} - e^{\eta^2/2}
            \frac{d}{d\eta^k} e^{-\eta ^ 2/2}
        \Big] 
        \\
    & =c(k)
        \Bigg[
            e^{x^2/2} \frac{d}{dx^k} e^{-x^2/2}
        \Big|_{x=\eta}^\xi\Bigg]
        \\
    & = c(k) \int_{\eta}^\xi F_k(x) dx,
\end{align*}
where $F_k$ is a continuous function such that
$F_k'(x)=e^{x^2/2}\frac{d}{dx^k}e^{-x^2/2}$. In fact, denoting by $Pe_k(x)$
the
unnormalized Hermite polynomial of $k$ degree, results
$$
    F_k'(x)=e^{x^2/2}\frac{d}{dx^k}e^{-x^2/2}= Pe_k(x),
$$
and since the Hermite polynomials constitute an Appell sequence we have that
$$
    F_k'(x)=Pe_k(x)=\frac{1}{k+1} Pe_{k+1}'(x),
$$
which implies that $F_k(x)=\frac{1}{k+1} Pe_{k+1}(x)$. Now, since $F_k(x)$ is
a continuous function, then there exists
$\gamma\in (\eta,\xi)$ such that
$$
    \int_{\eta}^\xi F_k(x) dx = F_k(\gamma)\cdot (\xi-\eta).
$$
 All these implies that
$
    P_k(\xi)-P_k(\eta)= c(k) F_k(\gamma) \cdot (\xi-\eta).
$
    From this expression the lemma follows immediately.
\end{proof}

Consider the stochastic differential equation in $\mathcal{H}$
\begin{equation}
\label{P1s2.1}
 dX_t=AX_tdt+B(X_t)dt+\sqrt{Q}dW_t,
\end{equation}
where the operator $A:\mathcal{D}(A)\subset \mathcal{H}\rightarrow
\mathcal{H}$ is the infinitesimal generator of a strongly
continuous semigroup $e^{tA}$ in $\mathcal{H}$, $Q$ is a bounded operator from
another Hilbert space $\mathcal{U}$ to $\mathcal{H}$
and $B:\mathcal{D}(B)\subset \mathcal{H}\rightarrow \mathcal{H}$ is a
nonlinear mapping.

The equation \eqref{P1s2.1} can be associated to a Kolmogorov equation in the
next way, we define
\begin{equation}
\label{P1s2.2}
u(t,x)=\E\big[\varphi(X_t^x)\big],
\end{equation}
where $\varphi:\mathcal{H}\rightarrow \IR$ and $X_t^x$ is the solution to
\eqref{P1s2.1} with initial conditions $X_0=x$ where
$x\in\mathcal{H}$. Then $u$ satisfies the Kolmogorov equation \eqref{P1s2.3}.
We will use some technical results on the SPDE to prove
the following result\textemdash the main result of this section.

\begin{theorem}
    \label{thm:ic_continuity}
    Assume that the eigenvalues of $\Lambda$, satisfies that for every 
    $k\in\IN$,
    $\lambda_k<\lambda_{k+1}\rightarrow \infty $. Assume that
    the functional $ \varphi$ is Lipschitz.
    Then, the numeric approximation $\Psi_t^\varphi$ (given by 
    \eqref{num-approx})
    to the solution of the Kolmogorov equation
    ${\Psi\in C((0,T); \mathbb{H})}$ also depends continuously on the initial
    conditions.
\end{theorem}
\begin{proof}
Let $x,y\in H$ be two different initial values. We want to estimate
$\Psi_t^x-\Psi_t^y$. By definition,
\begin{align}
    \Psi_t^x
        &= \sum_{\bar n\in \mathcal{J}}
            u_{\bar n} ^ x(t) H_{\bar n}(x) \ .
\end{align}
Thus,
%
\begin{equation}
    \label{s3.4}
    \begin{aligned}
        \Psi_t^x-\Psi_t^y
        &=
            \sum_{\bar n\in \mathcal{J}}
                u_{\bar n}^x(t)H_{\bar n}(x)
            -
            \sum_{\bar n\in \mathcal{J}}
                u_{\bar n}^y(t)H_{\bar n}(y)
       \\
        &=
            \sum_{\bar n\in \mathcal{J}}
                \Big[ u_{\bar n}^x(t )- u_{\bar n}^y(t)
                \Big]H_{\bar n}(x)
            +
            \sum_{\bar n\in \mathcal{J}}
                u_{\bar n}^y(t)
                \Big[
                    H_{\bar n}(x) - H_{\bar n}(y)
                \Big].
    \end{aligned}
\end{equation}

    We focus on the first term in \eqref{s3.4}. From the definition of the
initial condition we obtain the following expression for
the time-dependent coefficient
\begin{align*}
  u_{\bar n}^x(t) &=
    \int_{\mathcal{H}} H_{\bar n}(x) \E\big[\varphi(X_t^x)\big] \mu(dx) \ .
\end{align*}
%
From this we get
\begin{align*}
    u_{\bar n}^x(t)-u_{\bar n}^y(t)
        =&
        \int_{\mathcal{H}} H_{\bar n}(x)
            \E\big[
                \varphi(X_t^x)
            \big]
            \mu(dx) -
            \int_{\mathcal{H}} H_{\bar n}(y)
            \E\big[
                \varphi(X_t^y)
            \big] \mu(dy)
        \\
        =&
        \int_{\mathcal{H} \times \mathcal{H}} H_{\bar n}(x)
            \E\big[
                \varphi(X_t^x)
            \big] \mu(dx)\mu(dy)
        \\
        &-
             \int_{\mathcal{H}\times \mathcal{H}} H_{\bar n}(y)
                \E\big[
                    \varphi(X_t^y)
                \big]
            \mu(dx)\mu(dy)
        \\
        =&
        \int_{\mathcal{H}\times \mathcal{H}} H_{\bar n}(x)
            \Big(
                \E\big[
                    \varphi(X_t^x)
                \big]
                -
                \E\big[
                    \varphi(X_t^y)
                \big]
            \Big)
            \mu(dx)\mu(dy)
        \\
        &+
        \int_{\mathcal{H}\times \mathcal{H}}
            \Big(
                H_{\bar n}(x) - H_{\bar n}(y)
            \Big)
            \E\big[
                \varphi(X_t^y)
            \big] \mu(dx) \mu(dy) \ .
\end{align*}
%
%
%
Then, by the Cauchy-Schwartz inequality, we obtain
%
%
%
\begin{equation}
    \begin{aligned}
        |u_{\bar n} ^ x(t) - u_{\bar n} ^ y(t)| ^ 2
            \le&
            \left|
                \int_{\mathcal{H} 
                \times
                \mathcal{H}} H_{\bar n}(x)
                \left(
                    \E
                     \left[
                        \varphi(X_t^x)
                     \right]
                     -
                    \E\left[
                        \varphi(X_t^y)
                    \right]
                \right) \mu(dx)\mu(dy)
            \right|^2
            \\
            &+
            \left|
                \int_{\mathcal{H}
                \times
                \mathcal{H}
                }
                \left(
                    H_{\bar n}(x) - H_{\bar n}(y)
                \right)
                \E\left[
                    \varphi(X_t ^ y)
                \right]
                \mu(dx)\mu(dy)
            \right|^2
            \\
            \le &
            \int_{\mathcal{H}\times \mathcal{H}}
                H_{\bar n}^2(x))
            \mu(dx)\mu(dy)
            \\
            &\times
                \int_{\mathcal{H}\times \mathcal{H}}
                \left|
                    \E\left [
                        \varphi(X_t^x)
                    \right]
                    -
                    \E\left[
                        \varphi(X_t^y)
                    \right]
                \right|^2
                \mu(dx)\mu(dy)
            \\
            &+
            \int_{\mathcal{H} \times \mathcal{H}}
                \E^2
                \left[
                    \varphi(X_t^y)
                \right]
            \mu(dx)\mu(dy)
            \\
             &\times
            \int_{\mathcal{H}\times \mathcal{H}}
            \left|
                H_{\bar n}(x)-H_{\bar n}(y)
            \right|^2
            \mu(dx)\mu(dy)
            \\
            =&
            \int_{\mathcal{H}\times \mathcal{H}}
            \left|
                    \E 
                    \left [
                        \varphi(X_t^x)
                    \right]
                    -
                    \E\left[
                        \varphi(X_t^y)
                    \right]
                \right|^2
                \mu(dx) \mu(dy)
            \\
            &+
            \int_{\mathcal{H}\times \mathcal{H}}
                \E^2
                \left[
                    \varphi(X_t^y)
                \right]
            \mu(dx)\mu(dy)
            \\
            & \times
            \int_{\mathcal{H}\times \mathcal{H}}
            \left|
                H_{\bar n}(x)
                    -
                    H_{\bar n}(y)
            \right|^2
            \mu(dx)\mu(dy)
        \ .
    \end{aligned}
    \label{s3.4.1}
\end{equation}
We now estimate the norm of the expression \eqref{s3.4} with the help of
\eqref{s3.4.1}.
%
\begin{equation}
\label{s3.8.1}
    \begin{aligned}
        \| \Psi_t ^ x - \Psi_t ^ y & \|_{
            \big(
                L^2(\mathcal{H},\mu)
            \big)^2
        }^2
            =
                \int_{\mathcal{H}\times \mathcal{H}}
                |\Psi_t^x - \Psi_t^y|^2 \mu(dx) \mu(dy)
            \\
            \le&
            \int_{\mathcal{H} \times \mathcal{H}}
                \Big|
                    \sum_{\bar n\in \mathcal{J}}
                    \big[
                        u_{\bar n} ^ x(t) - u_{\bar n} ^ y(t)
                    \big]
                    H_{\bar n}(x)
                \Big|^2
                \mu(dx) \mu(dy)
            \\
            &+
            \int_{\mathcal{H}\times \mathcal{H}}
            \Big|
                \sum_{\bar n\in \mathcal{J}}
                    u_{\bar n}^y(t)
                    \big[
                        H_{\bar n}(x) - H_{\bar n}(y)
                    \big]
            \Big|^2
            \mu(dx) \mu(dy)
            \\
            \le&
            \int_{\mathcal{H} \times \mathcal{H}}
            \sum_{\bar n\in \mathcal{J}}
            \big|
                u_{\bar n}^x(t) - u_{\bar n}^y(t)
            \big |^2
            H_{\bar n}^2(x)
            \mu(dx) \mu(dy)
            \\
            &+
            \int_{\mathcal{H}\times \mathcal{H}}
            \sum_{\bar n\in \mathcal{J}}
            \big[u_{\bar n} ^ y(t) \big]^2
            \sum_{\bar n\in \mathcal{J}}
            \big|
                H_{\bar n}(x) - H_{\bar n}(y)
            \big|^2
            \mu(dx) \mu(dy)
            \\
            =&
            \sum_{\bar n\in \mathcal{J}}
            \big|
                u_{\bar n}^x(t)-u_{\bar n}^y(t)
            \big |^2
            \int_{\mathcal{H} \times \mathcal{H}}
            H_{\bar n}^2 (x)
            \mu(dx) \mu(dy)
            \\
            &+
            \sum_{\bar n\in \mathcal{J}}
            \big[
                u_{\bar n}^y(t)
            \big] ^ 2
            \sum_{\bar n\in \mathcal{J}}
            \int_{\mathcal{H}\times \mathcal{H}}
            \big|
                H_{\bar n}(x) - H_{\bar n}(y)
            \big| ^ 2 \mu(dx)\mu(dy)
            \\
            =&
            \sum_{\bar n\in \mathcal{J}}
            \big[
                u_{\bar n}^y(t)
            \big] ^ 2
            \sum_{\bar n\in \mathcal{J}}
            \int_{\mathcal{H}\times \mathcal{H}}
            \big|
                H_{\bar n}(x) -H_{\bar n}(y)
            \big|^2 \mu(dx) \mu(dy)
            \\
            &+
            \sum_{\bar n\in \mathcal{J}}
                        \big|
                            u_{\bar n}^x(t)-u_{\bar n}^y(t)
                        \big| ^ 2
            \\
            =&
            \int_{\mathcal{H}\times \mathcal{H}}
            \Big|
                \E\big[
                    \varphi(X_t^x)
                \big]
                -
                \E\big[
                    \varphi(X_t^y)
                \big]
            \Big|^2
            \mu(dx)\mu(dy)
            \\
            &+
            \int_{\mathcal{H}\times \mathcal{H}}
            \E ^ 2\big[
                \varphi(X_t^y)
            \big] \mu(dx) \mu(dy)
            \\
            & \times
            \sum_{\bar n\in \mathcal{J}}
            \int_{\mathcal{H}\times \mathcal{H}}
            \Big|
                H_{\bar n}(x) - H_{\bar n}(y)
            \Big|^2  \mu(dx)\mu(dy)
            \\
            &+
            \sum_{\bar n\in \mathcal{J}}
            \big[
                u_{\bar n}^y(t)
            \big]^2
            \sum_{\bar n\in \mathcal{J}}
            \int_{\mathcal{H}\times \mathcal{H}}
            \big|
                H_{\bar n}(x) - H_{\bar n}(y)
            \big|^2 \mu(dx)\mu(dy)
            \ .
    \end{aligned}
\end{equation}

    Notice that
$\E^2\big[\varphi(X_t^y)\big]=u^2(t,x)\in L^2(\mathcal{H},\mu)$,
therefore the first integral in the second term is a continuous bounded
function of $t$. Moreover,
$\sum_{\bar n\in \mathcal{J}} \big[u_{\bar n}^y(t)\big]^2$ is the
$L^2(\mathcal{H},\mu)$-norm of the function $u(t,x)$, then the series
converges and it is also a continuous bounded function of $t$. Thus, from
\eqref{s3.8.1} we get
%
\begin{equation}
\label{s3.9.1}
    \begin{aligned}
        \| \Psi_t^x-\Psi_t^y\|_{\big(L^2(\mathcal{H},\mu) \big)^2}^2
        \le&
        \int_{\mathcal{H}\times \mathcal{H}}
            \Big|
                \E\big [
                    \varphi(X_t^x)
                \big
                ]-
                \E\big[
                    \varphi(X_t^y)
                \big]
            \Big|^2
        \mu(dx)\mu(dy)
        \\
         &+
         f(t)
         \sum_{\bar n\in \mathcal{J}}
         \int_{\mathcal{H}\times \mathcal{H}}
            \big|
                H_{\bar n}(x) - H_{\bar n}(y)
            \big|^2 \mu(dx)\mu(dy) \ ,
    \end{aligned}
\end{equation}
where $f(t)= \sum_{\bar n\in \mathcal{J}} \big[u_{\bar n}^y(t)\big]^2$.

From the proof of Theorem \ref{Cont-Mild-Sol} (see \eqref{s2.23}) we know that
\begin{equation}
    \begin{aligned}
        \| | \Psi_t^\varphi-\Phi_t^\psi\| |^2
            &=\int_{\mathcal{H}\times \mathcal{H}}
            \Big|
                \E\big [
                    \varphi(X_t^x)
                \big]
                -
                \E\big[
                    \varphi(X_t^y)
                \big]
            \Big|^2
            \mu(dx)
            \mu(dy)
            \\
            &\le
            \exp(Ct)
            \int_{\mathcal{H}\times \mathcal{H}}
                \|x-y\|_{\mathcal{H}}^2
                \mu(dx)\mu(dy)
            \\
            &= \exp(Ct) \|  |x-y\| |^2.\label{s3.10.1}
    \end{aligned}
\end{equation}
Therefore the first term in the right side of \eqref{s3.9.1} is bounded by
\eqref{s3.10.1}.

We now focus on the second term in the last inequality. Notice that for every
$\bar n\in \mathcal{J}$ we have
\begin{align}\label{s3.5}
    H_{\bar n}(x) - H_{\bar n}(y) =
        \prod_{i = 1} ^ \infty
    \Big[
        P_{n_i}( \xi_i ) - P_{n_i} (\eta_i)
    \Big],
\end{align}
where $\xi_i= \langle x,\Lambda^{-1/2}e_i\rangle_{\mathcal{H}}$ and $\eta_i=
\langle y,\Lambda^{-1/2}e_i\rangle_{\mathcal{H}}$
(see \eqref{s1.2} and lines after that for the definition).
Hence, applying \Cref{le-s3-1} to equation \eqref{s3.5} we have that

\begin{equation}
\label{s3.6}
\begin{aligned}
    H_{\bar n}(x)-H_{\bar n}(y)&=\prod_{i=1}^\infty C(i)Pe_{i+1}(\gamma_i)
    \cdot
    (\xi_i-\eta_i)
    \\
    &=
    \prod_{i=1}^\infty
    C(i)Pe_{i+1}(\gamma_i)
    \langle
        x-y, \Lambda^{-1/2}e_i
    \rangle_{\mathcal{H}},
 \end{aligned}
\end{equation}
here $\gamma_i\in \big(\xi_i\wedge\eta_i, \xi_i\vee \eta_i  \big)$
for every $i\in \IN$.
Then
\begin{equation}
    \label{s3.8}
    \begin{aligned}
        \sum_{\bar n\in \mathcal{J}}
        &
        \int_{\mathcal{H}\times \mathcal{H}}
            \big|
                H_{\bar n}(x) -H_{\bar n}(y)
            \big|^2
        \mu(dx)\mu(dy)
        \\
        &=
        \sum_{\bar n\in \mathcal{J}}
            \int_{\mathcal{H}\times \mathcal{H}}
                \Big|
                    \prod_{i=1}^\infty C(i)Pe_{i+1}(\gamma_i)
                    \langle
                        x-y,
                        \Lambda^{-1/2}e_i
                    \rangle_{\mathcal{H}}
                \Big|^2
        \mu(dx)\mu(dy)
        \\
        &=
        \sum_{\bar n\in \mathcal{J}}
        \int_{\mathcal{H}\times \mathcal{H}}
            \prod_{i=1}^\infty
            \Big[
                C(i) Pe_{i+1}(\gamma_i)\Big]^2
            \Big|
                \langle
                    x-y,\Lambda^{-1/2}e_i
                \rangle_{\mathcal{H}}
            \Big|^2
            \mu(dx) \mu(dy)
        \\
        &\le
            \sum_{\bar n\in \mathcal{J}}
            \int_{\mathcal{H} \times \mathcal{H}}
                \prod_{i=1}^\infty
                \Big[
                    C(i)Pe_{i+1}(\gamma_i)
                \Big]^2
                \| x-y\|_{\mathcal{H}}^2
                \|\Lambda^{-1/2} e_i\|_{\mathcal{H}}^2
            \mu(dx)\mu(dy)
        \\
        &=
            \sum_{\bar n\in \mathcal{J}}
            \int_{\mathcal{H}\times \mathcal{H}}
                \prod_{i=1}^\infty
                \Big[
                    C(i)Pe_{i+1}(\gamma_i)
                \Big]^2
                \| x-y\|_{\mathcal{H}}^2
                \lambda_i^{-1}
                \|e_i\|_{\mathcal{H}}^2
            \mu(dx)\mu(dy)
        \\
        &=
        \| x-y\|_{\mathcal{H}}^2
        \sum_{\bar n\in \mathcal{J}}
            \prod_{i=1}^\infty
                \Big[
                    C(i)
                \Big]^2
                \lambda_i^{-1}
                \int_{\mathcal{H}\times \mathcal{H}}
                    \Big[
                        Pe_{i+1}(\gamma_i)
                    \Big]^2
                \mu(dx)\mu(dy) .
    \end{aligned}
\end{equation}
Recall that for every $i \in \IN$ we have that 
$
    \gamma_i\in
    \big(
        \xi_i \wedge\eta_i, \xi_i\vee \eta_i  
   \big)
$, set 
$
    \hat \gamma_i
    \in 
    \big(
        \xi_i \wedge \eta_i,
        \xi_i\vee \eta_i  
    \big)
$ such that
$
    Pe_i^2(\gamma_i)
    \le Pe_{i+1}^2(\hat\gamma_i)
$ for every
$
    \gamma_i
    \in 
    \big(
        \xi_i \wedge\eta_i, 
        \xi_i \vee
        \eta_i
    \big)
$, notice that the
existence of $\hat \gamma_i$ is
guaranteed since $Pe_{i+1}^2(\cdot)$ 
is a continuous function. Then, from
\eqref{s3.8} we get
\begin{equation}
    \label{s3.9}
    \begin{aligned}
        \sum_{\bar n\in \mathcal{J}}
        \int_{\mathcal{H}\times \mathcal{H}}
        &
        \big|
            H_{\bar n}(x) - H_{\bar n}(y)
        \big|^2
        \mu(dx)\mu(dy)
        \\
        &\le
        \| x-y\|_{\mathcal{H}}^2
        \sum_{\bar n\in \mathcal{J}}
            \prod_{i=1}^\infty
            \Big[
                C(i)
            \Big]^2
            \lambda_i^{-1}
            \Big[
                Pe_{i+1}(\hat \gamma_i)
            \Big]^2
            \int_{\mathcal{H}}
            \int_{\mathcal{H}}
            \mu(dx)\mu(dy)
        \\
        &=
        \| x-y\|_{\mathcal{H}}^2
        \sum_{\bar n\in \mathcal{J}}
            \prod_{i=1}^\infty
            \Big[
                C(i)
            \Big]^2
            \lambda_i^{-1}
            \Big[
                Pe_{i+1}(\hat \gamma_i)
            \Big]^2 .
    \end{aligned}
\end{equation}
Here, we recall that $C(i)=\frac{(-1)^i}{(i+1)(i!)^{1/2}} $ then
$\frac{(-1)^i}{\big[(i+1)!\Big]^{1/2}} Pe_{i+1}(\hat \gamma_i)$
is the normalized Hermite polynomial of $i+1$ degree evaluated on
$\hat \gamma_i$ which is bounded by a constant $C$ for every $i\in\IN$.
Moreover, since $\lambda_k<\lambda_{k+1}\rightarrow \infty $
then this implies that
\begin{align}
    \label{s3.10}
  \sum_{\bar n\in \mathcal{J}}  \prod_{i=1}^\infty \Big[C(i)\Big]^2
  \lambda_i^{-1} \Big[Pe_{i+1}(\hat \gamma_i)\Big]^2 & \le C \sum_{\bar n\in
    \mathcal{J}}  \prod_{i=1}^\infty
  \lambda_i^{-1}(i+1)^{-1}
  \le C,
\end{align}
where $C$ is a finite constant. Putting together \eqref{s3.8} and
\eqref{s3.10} we get that
\begin{align}
    \sum_{\bar n\in \mathcal{J}} \int_{\mathcal{H}\times \mathcal{H}}  \big|
    H_{\bar n}(x) -H_{\bar n}(y) \big|^2 \mu(dx)\mu(dy) &\le
    C \| x-y\|_{\mathcal{H}} \ .
    \label{s3.11}
\end{align}
Putting together inequalities \eqref{s3.9.1}, \eqref{s3.10.1} and
\eqref{s3.11} we obtain
\begin{align}
  \| \Psi_t^x-\Psi_t^y\|_{\big(L^2(\mathcal{H},\mu) \big)^2}^2&\le   \exp(Ct)
  \int_{\mathcal{H}\times \mathcal{H}}
  \| x-y\|_{\mathcal{H}}^2 \mu(dx)\mu(dy) +  f(t)\| x-y\|_{\mathcal{H}}.
  \label{s3.17}
\end{align}
Now, if $\| x-y\|_{\mathcal{H}}\le \delta $, then from \eqref{s3.17} we get
$
 \| \Psi_t^x-\Psi_t^y\|_{\big(L^2(\mathcal{H},\mu) \big)^2} \le
  G(t) \delta %\label{s3.12}
$.
\end{proof}

\begin{remark}
    If we consider in addition the supremum norm on $t$, then from
    \eqref{s3.17} we get
    \begin{equation}
        \label{s3.13}
        \begin{aligned}
            \sup_{0\le t\le T}\| \Psi_t^x-\Psi_t^y\|_{
                \big(L^2(\mathcal{H},\mu) \big)^2} ^ 2
                \le&
                 C \| x-y\|_{\mathcal{H}}^2
                 \sup_{0\le t\le T} f(t)
                 \\
                 &+
                \exp(CT)
                \int_{\mathcal{H}\times \mathcal{H}}
                \|x-y\|_{\mathcal{H}}^2 \mu(dx)\mu(dy) \ .
        \end{aligned}
    \end{equation}
    Notice that $f(t)$ is differentiable and continuous, then
    $\sup_{0\le t\le T} f(t)\le C$, then from \eqref{s3.13} we obtain
    \begin{equation}
        \label{s3.14}
        \begin{aligned}
            \sup_{0\le t\le T}
            \|
                \Psi_t^x-\Psi_t^y
            \|_{\big(L^2(\mathcal{H},\mu)\big)^2}
            &\le
            C \| x-y\|_{\mathcal{H}}
            \\
            & +
            \exp(CT)
            \int_{\mathcal{H}\times \mathcal{H}}
            \| x-y\|_{\mathcal{H}}^2 \mu(dx)\mu(dy) \ .
        \end{aligned}
    \end{equation}
    From this inequality it is possible to show the continuously dependence on
    the initial conditions for this norm.
\end{remark}

\section{Numerical experiments}
    \label{sec:NumericalExperiments}
        In this section we run numerical experiments to illustrate that our
    scheme preserves the underlying initial condition continuity. To 
    this end, we solve  a stochastic version of the  Fisher and Burgers PDEs 
    with two near initial function conditions $x(\xi)$, 
    $\widehat{x}(\xi)$. In \cite{matsumyaRepo} we provide a GitHub repository 
    with a Python implementation to reproduce the following figures. We also 
    provide in \cite{plotlyFisher, plotlyBurgers}, the 3D on-line plotly 
    versions of \Cref{fig:likening_fisher_kpp,fig:likening_burgers}.
%
    \subsection*{Stochastic Fisher-KPP equation in an interval}
        Let $\mathcal{H} = L^2 (0,1)$. We consider the stochastic Fisher-KPP 
    equation in the interval $[0, 1]$
    \begin{equation}
        \label{eqn:fisher-kpp}
        \begin{aligned}
            d X(t, \xi) &= 
                \left[
                    \nu 
                    \partial_{\xi} ^ 2 X(t, \xi)
                    +
                    X(t, \xi) (1 -X(t, \xi) )
                \right]
                dt
                +
                dW(t, \xi),
            \\
            X(t, 0) &= X(t, 1) =0, \quad t>0, 
            \\
            X(0, \xi) &= x(\xi), \quad x\in \mathcal{H}
        \end{aligned}
    \end{equation}
    
\begin{table}[H]
    \begin{tabular}{rc}
        \toprule
        $h$ 
        &
        $x$
        \\
        \bottomrule
    \end{tabular}
\end{table}
%
\begin{figure}[H]
    \centering
    \caption{
        Numerical Solution of the Fisher-KPP 
        \cref{eqn:fisher-kpp} 
        with initial conditions $x$, $y$ at time
        $t=0$.
     }
    \label{fig:fisher_kpp_approximation_t0}
    \includegraphics[width=\linewidth, keepaspectratio]%
    {StochasticFisherEquation/Approximation_t=0}
\end{figure}
%
\begin{figure}[H]
    \centering
    \caption{
        Likening between two solution with closed 
        initial conditions $x$, $y$
        of the stochastic Fisher-KPP
        \cref{eqn:fisher-kpp}.
     }
    \label{fig:likening_fisher_kpp}
    \includegraphics[width=\linewidth, keepaspectratio]%
    {StochasticFisherEquation/simulation_Approximation.png}
\end{figure}
\paragraph{Likening}
    In \Cref{fig:likening_fisher_kpp} we illustrate the distance between initial
    conditions distance between...
\todo{A paragraph to describe and stress the likening}
\todo{plot the two solutions and make zooms to stress the contrast}
%
\todo{Change linear scale to logy scale}
    \subsection*{Stochastic Burgers equation}
        Let $\mathcal{H} = L^2(0,1)$, consider the stochastic Burgers equation in the 
interval $[0, 1]$
\begin{equation}
    \label{eqn:stochastic_burgers}
    \begin{aligned}
        d X(t, \xi) &= 
            \left[
                \nu \partial_{\xi} ^ 2 X(t, \xi)
                + \frac{1}{2} \partial_{\xi} X^2(t, \xi)
            \right]dt
            +dW(t, \xi),
            \\
        X(t, 0) &= X(t, 1) =0, \quad t>0, \\
        X(0, \xi) &= x(\xi), \quad x\in \mathcal{H} \ .
    \end{aligned}
\end{equation} 
As in the above experiment, we use the initial conditions $x(\xi)$ and its 
truncated Chebyshev expansion 
\begin{equation}
    x(\xi) := \sin(\pi \xi),
    \qquad
    \widehat{x}(\xi) :=
        \sum_{k=0} ^ N
         T_k x(\xi).
\end{equation}

\Cref{fig:approximationt0,fig:likening_burgers,fig:error_convergence} 
illustrate a similar argument presented  in the above experiment.
\begin{figure}[H]
    \caption{
        Numerical Solution of the Burgers 
        \cref{eqn:stochastic_burgers} 
        with initial conditions $x(\xi)$, $\widehat{x}(\xi)$.
     }
    \label{fig:approximationt0}
    \includegraphics[width=\linewidth, keepaspectratio]%
    {StochasticBurgersEquation/Approximation_t=0.eps}
\end{figure}

\begin{figure}[H]
    \centering
    \caption{
        Likening between two solution with closed 
        initial conditions $x(\xi)$, and $\widehat{x}(\xi)$
        of the stochastic Burgers
        \cref{eqn:stochastic_burgers}.
     }
    \label{fig:likening_burgers}
    \includegraphics[width=\linewidth, keepaspectratio]%
    {StochasticBurgersEquation/simulation_Approximation.png}
\end{figure}
%
\begin{figure}[H]
    \centering
    \caption{
        Distance between two solutions of the
        stochastic Burgers
        \cref{eqn:stochastic_burgers}
        with initial conditions  $x = x(\xi)$, and $y = x(\xi)$.
     }
    \label{fig:error_convergence}
    \includegraphics[width=\linewidth, keepaspectratio]%
    {StochasticBurgersEquation/error_burgers.eps}
\end{figure}





    \bibliographystyle{siamplain}
    \bibliography{references}
\end{document}
