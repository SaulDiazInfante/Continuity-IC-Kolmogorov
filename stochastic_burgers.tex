Let $\mathcal{H} = L^2(0,1)$, consider the stochastic Burgers equation in the 
interval $[0, 1]$
\begin{equation}
    \label{eqn:stochastic_burgers}
    \begin{aligned}
        d X(t, \xi) &= 
            \left[
                \nu \partial_{\xi} ^ 2 X(t, \xi)
                + \frac{1}{2} \partial_{\xi} X^2(t, \xi)
            \right]dt
            +dW(t, \xi),
            \\
        X(t, 0) &= X(t, 1) =0, \quad t>0, \\
        X(0, \xi) &= x(\xi), \quad x\in \mathcal{H} \ .
    \end{aligned}
\end{equation} 
As in the above experiment, we use the initial conditions $x(\xi)$ and its 
truncated Chebyshev expansion 
\begin{equation}
    x(\xi) := \sin(\pi \xi),
    \qquad
    \widehat{x}(\xi) :=
        \sum_{k=0} ^ N
         T_k x(\xi).
\end{equation}

\Cref{fig:approximationt0,fig:likening_burgers,fig:error_convergence} 
illustrate a similar argument presented  in the above experiment.
\begin{figure}[H]
    \caption{
        Numerical Solution of the Burgers 
        \cref{eqn:stochastic_burgers} 
        with initial conditions $x(\xi)$, $\widehat{x}(\xi)$.
     }
    \label{fig:approximationt0}
    \includegraphics[width=\linewidth, keepaspectratio]%
    {StochasticBurgersEquation/Approximation_t=0.eps}
\end{figure}

\begin{figure}[H]
    \centering
    \caption{
        Likening between two solution with closed 
        initial conditions $x(\xi)$, and $\widehat{x}(\xi)$
        of the stochastic Burgers
        \cref{eqn:stochastic_burgers}.
     }
    \label{fig:likening_burgers}
    \includegraphics[width=.9\textwidth, keepaspectratio]%
    {StochasticBurgersEquation/simulation_Approximation.png}
\end{figure}
%
\begin{figure}[H]
    \centering
    \caption{
        Distance between two solutions of the
        stochastic Burgers
        \cref{eqn:stochastic_burgers}
        with initial conditions  $x = x(\xi)$, and $y = \widehat{x}(\xi)$.
     }
    \label{fig:error_convergence}
    \includegraphics[width=\linewidth, keepaspectratio]%
    {StochasticBurgersEquation/error_burgers.eps}
\end{figure}




